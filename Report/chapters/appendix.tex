
\فصل{مطالب تکمیلی}

\definecolor{dkgreen}{rgb}{0,0.6,0}
\definecolor{gray}{rgb}{0.5,0.5,0.5}
\definecolor{mauve}{rgb}{0.58,0,0.82}

\lstset{frame=tb,
	language=python,
	aboveskip=3mm,
	belowskip=3mm,
	lineskip=1pt,
	showstringspaces=false,
	columns=flexible,
	basicstyle={\small\ttfamily},
	numbers=left,
	numberstyle=\tiny\color{gray},
	keywordstyle=\color{blue},
	commentstyle=\color{dkgreen},
	stringstyle=\color{mauve},
	breaklines=true,
	breakatwhitespace=true,
	tabsize=3,
}

در این قسمت، بخشی از کد های مهم پروژه آورده شده است. فایل‌های اصلی پروژه را می‌توانید از این لینک\زیرنویس{ لینک دریافت: \href{https://github.com/rezaAdinepour/Bachelors-Project}{\کد{github.com/rezaAdinepour/Bachelors-Project}}} دریافت کنید.

\قسمت{استخراج بردار ویژگی‌های LBP}
در این قسمت برنامه ای ساده برای استخراج ویژگی های LBP و تشکیل بردار ویژگی‌ها نوشته شده است. \\
• برنامه استخراج ویژگی‌های LBP
\begin{latin}
	\lstinputlisting[language=Python]{Code/LBP/LBP_Features.py}
\end{latin}

\قسمت{تست دوربین}
در این قسمت ابتدا از اتصال دوربین به بورد اطمینان حاصل می‌کنیم. \\
• برنامه اتصال دوربین به بورد
\begin{latin}
	\lstinputlisting[language=Python]{Code/CamTest.py}
\end{latin}

\قسمت{فاز آموزش}
• تابع نوشته شده در بخش آموزش برنامه به صورت زیر است:
\begin{latin}
	\lstinputlisting[language=Python]{Code/train_phase_function.py}
\end{latin}


\قسمت{فاز تست و شناسایی چهره}
در بحث شناسایی چهره، مهم‌ترین گام تشخصی محل چهره در تصویر است که در این قسمت آن را تست کردیم.

• برنامه تشخیص چهره
\begin{latin}
	\lstinputlisting[language=Python]{Code/FaceDetection.py}
\end{latin}