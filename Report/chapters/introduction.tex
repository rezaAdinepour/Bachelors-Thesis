%---------------------------------------------------------------------------------------------
%‌ Chapter 1
%---------------------------------------------------------------------------------------------
\فصل{مقدمه}

سیستم تشخیص و شناسایی چهره\پاورقی {Face Detect and Recognition System} یک تکنولوژی توانا در شناسایی و تأیید هویت یک فرد از یک عکس دیجیتالی یا ویدئو می‌باشد. سیستم تشخیص چهره، سیستمی است که بر اساس تکنولوژی هوش مصنوعی\پاورقی {Artifitial Intelligence} و قادر به شناسایی چهره افراد با دقت بالا می‌باشد. در بازشناخت تصویر یک چهره، تصویر ورودی با توجه به اطلاعات موجود در بانک اطلاعات، مورد شناسایی قرار می‌گیرد. این بانک شامل مشخصاتی از تصویر چهره افراد شناسایی شده‌است. 
%---------------------------------------------------------------------------------------------
\قسمت{تعریف مسئله}

در بسیاری از کاربردها نیازمند شناسایی چهره افراد هستیم.
در مقیاس های کوچک شاید بتوان از ناظر انسانی جهت انجام این کار استفاده کرد.
اما با گسترش مقیاس مسئله، انجام این ‌کار توسط انسان قدری دشوار خواهد بود.
در این نقطه کامپیوتر به کمک انسان می‌آید و انجام کار را آسان می‌کند.
همکاری ای که بینایی ماشین\پاورقی {Computer Vision} نامیده میشود.
%---------------------------------------------------------------------------------------------
\قسمت{اهمیت موضوع}

از گذشته تا به امروز، مسئله شناسایی چهره از اهمیت زیادی برخوردار بوده است.
چرا که با شناسایی افراد، هویت آنها فاش میشود. محل های زیادی وجود دارد که 
شناسایی چهره در آنجا اهمیت ویژه ای دارد مانند:

\شروع{فقرات}
\فقره
فرودگاه‌ها
\فقره
اداره‌ها
\فقره
دانشگاه‌ها
\فقره
مکان‌های عمومی
\پایان{فقرات}
%---------------------------------------------------------------------------------------------
\قسمت{اهداف پروژه}

هدف از انجام این پایان‌نامه، طراحی و ساخت سیستمی قابل حمل\پاورقی {Portable}،
بلادرنگ\پاورقی {Real Time} با قابلیت اطمینان بالا به کمک هوش مصنوعی و طبقه‌بند های مختلف بوده است که تا حد زیادی انتظارات ما در انجام این پروژه برآورده شده است.
%---------------------------------------------------------------------------------------------
\قسمت{ساختار پایان‌نامه}

این پایان‌نامه در 5 فصل و 1 پیوست به شرح زیر ارائه می‌شود:

فصل دوم به معرفی ساختار طبقه‌بند‌ های مهم همچون K-NN ، آبشاری، کاربرد ها و ویژگی های آن‌ها می‌پردازد.
فصل سوم به معرفی بُرد ،Odroid نصب سیستم عامل و پکیج های مورد نیاز برای انجان پروژه اختصاص دارد.
در فصل چهارم، پیاده سازی سیستم و نتایج بدست آمده از انجام این پروژه، ارائه می‌شود. فصل پنجم به جمع‌بندی کارهای انجام شده در این پروژه و ارائه‌ی پیشنهادهایی برای انجام
کارهای آتی خواهد پرداخت. در پیوست شماره 1، برخی از کدهای مهم استفاده شده در پروژه آورده شده است\زیرنویس{ تمامی کد‌ها از این لینک قابل دریافت است: \href{https://github.com/rezaAdinepour/Bachelors-Project}{\کد{github.com/rezaAdinepour/Bachelors-Project}}}.
 %---------------------------------------------------------------------------------------------