
\فصل{نتیجه‌گیری}

در این پایان‌نامه به برسی و پیاده سازی سیستمی بلادرنگ برای تشخیص چهره مبتنی بر ویژگی‌های LBP و طبقه‌بند آبشاری پرداخته شد که در فصل های 1، 2 به معرفی ساختار طبقه‌بند ها و تئوری مورد نیاز پایان‌نامه پرداخته شد. در فصل 3 سخت‌افزار استفاده شده برای ساخت معرفی شد و نحوه راه‌اندازی آن برسی شد و در فصل 4 نتایج بدست آمده گزارش شده است.

با گسترش تکنولوژی و هوش‌مصنوعی، امید است سیستم‌های سنتی حضور و غیاب در ادارت، سیستم‌های شناسایی چهره در محیط‌های عمومی و ... دسخوش تغییراتی بشوند، از این رو در این پایان‌نامه در جهت حرکت به این‌سو گام برداشتیم. سیستم طراحی شده در این پایان‌نامه می‌تواند جایگزین مناسبی برای سیستم حضور و غیاب در اداره ها و دانشگاه ها باشد.

هسته اصلی سخت‌افزار این پایان‌نامه مینی کامپیوتر Odroid است که به دلیل سهولت در جا‌به‌جایی از آن استفاده شده است. دلیل دوم استفاده از این بُرد سرعت پردازش بالای آن در مقایسه با بُرد های هم‌رده همچون pi Raspberry است.

در پیاده‌سازی این پروژه از یک عدد Webcam و مانیتور استفاده شده است. استفاده از مانیتور های معمولی سهولت در جا‌به‌جایی و Portable بودن سیستم را از بین می‌برد از این جهت پیشنهاد می‌شود از مانیتور های کوچک با قابلیت تاچ خازنی استفاده شود تا به طور کامل Portable بودن سیستم تضمین شود.

و در نهایت از تمام کسانی که این پایان‌نامه را مطالعه کردند عرض تشکر دارم و درخواست دارم تا نقطه‌نظراتون را با ایمیل زیر با بنده در‌میان بگذارید: \\
\begin{latin}
	\href{mailto:reza_adinepour@shahroodut.ac.ir}{\کد{reza\_adinepour@shahroodut.ac.ir}}
\end{latin}

