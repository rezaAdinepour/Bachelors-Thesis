
\فصل{اجرای الگوریتم و نتایج}

در این فصل به برسی الگوریتم نوشته شده و اجرای آن و برسی خروجی‌ها می‌پردازیم. هسته اصلی کد از این لینک\زیرنویس{ \href{https://github.com/kunalyelne/Face-Recognition-using-Raspberry-Pi}{\کد{github.com/kunalyelne/Face-Recognition-using-Raspberry-Pi}}} دانلود شده و آن را دستخوش تغییرات دادیم و برای آن رابط کاربری نوشتیم.

\قسمت{رابط کاربری برنامه}
برای الگوریتم نوشته شده، یک برنامه\پاورقی {Application} به زبان پایتون «بخشی از کد‌ ها در پیوست آ،  آورده شده» با رابط کاربری آسان نوشته شده است. «شکل \رجوع{شکل:رابط کاربری برنامه}»


\شروع{شکل}[h]
\centerimg{Attendance_GUI}{12cm}
\شرح{رابط کاربری برنامه}
\برچسب{شکل:رابط کاربری برنامه}
\پایان{شکل}

این برنامه از 3 بخش کلی تشکیل شده است:

\شروع{شمارش}

\فقره
\مهم{بخش اول: }وارد کردن ID و نام کاربر جهت تهیه عکس از فرد و ذخیره آن در دیتاست برنامه.

\فقره
\مهم{بخش دوم: }آموزش شبکه بر روی دیتاست تهیه شده.

\فقره
\مهم{بخش سوم: }تست شبکه آموزش دیده.

\پایان{شمارش}

با وارد کردن ID و نام کاربر و زدن دکمه Image ،Take دوربین متصل به بُرد فعال شده و 100 عدد عکس از کاربر می‌گیرد\زیرنویس{ از این لینک می‌توانید دیتاست را دانلود کنید: \href{https://drive.google.com/drive/folders/14Zx6iGrpOwKMenNNCljpov8-e56zhVn8?usp=sharing}{\کد{b2n.ir/dataset}}}
« شکل \رجوع{شکل:تهیه دیتاست}» و با نام های \کد{User.Id.i} که به ترتیب \کد{User} و \کد{Id} اسم و آیدی وارد شده و \کد{i} شماره عکس است. ضمن اینکه تمامی ID ها و Username های وارد شده در فایل اکسل با نام \کد{StudentDetails} ذخیره می‌شوند. «شکل \رجوع{شکل:ذخیره اطلاعات کاربران}»

\شروع{شکل}[h]
\centerimg{Attendance_Take_Img}{12cm}
\شرح{تهیه دیتاست}
\برچسب{شکل:تهیه دیتاست}
\پایان{شکل}

\شروع{شکل}[h]
\centerimg{Attendance_Save_User}{12cm}
\شرح{اطلاعات ذخیره شده کاربران}
\برچسب{شکل:ذخیره اطلاعات کاربران}
\پایان{شکل}

پس از آنکه عکس ها گرفته شد، با استفاده از دکمه Profile ،Save می‌توان شبکه را Train کرد. با زدن این دکمه باکسی نمایش داده می‌شود که باید Password سیستم را وارد کرد. به صورت پیش‌فرض، پسورد سیستم \کد{9814303} در نظر گرفته شده است. با وارد کردن پسورد و زدن کلید ،Ok شبکه Train می‌شود.

\قسمت{فاز آموزش شبکه}
پس از تهیه دیتاست، با زدن دکمه Profile ،Save باید شبکه را آموزش داد. با وارد کردن Password برنامه، این Password در تابعی به نام \کد{psw} با مقدار ذخیره شده آن در تنظیمات برنامه مقایسه می‌شود و اگر مقدار آن درست بود وارد تابع \کد{TrainImages} می‌شود و فرایند آموزش آغاز می‌شود.

در ابتدای ورود به تابع \کد{TrainImages} چک می‌شود که آیا فایل ضرایب Haar که در فایلی با نام \کد{haarcascade\_frontalface\_default.xml} قرار دارد، وجود دارد یا خیر.
سپس از وجود مسیر\پاورقی{Directory} جهت ذخیره ضرایب آموزش دیده اطمینان حاصل می‌کنیم.
سپس با این دستور \کد{cv2.face\_LBPHFaceRecognizer.create()} یک شیء براش شناسایی چهره ها می‌سازیم. سپس همه عکس‌های و آیدی‌های موجود در پوشه دیتاست با نام \کد{TrainingImage} را به ترتیب در تابعی با نام \کد{getImagesAndLabels} به شبکه می‌دهیم و با ماژول \کد{train} ضرایب مطلوب را بدست می‌اوریم و در فایلی با نام \کد{Trainner.yml} ذخیره می‌کنیم.


\قسمت{فاز تست و شناسایی چهره‌ها}
پس از آنکه آموزش شبکه به اتمام رسید، نوبت به تست شبکه می‌رسد. با زدن دکمه Attendance ،Take فاز شناسایی چهره های از پیش آموزش داده شروع می‌شود. «شکل \رجوع{شکل:تست سخت}» این فاز در تابعی به نام \کد{TrackImages} انجام شده است. در ابتدای کار از وجود مسیر‌های ذخیره‌سازی افراد شناسایی شده و اطلاعات دیتاست اطمینان حاصل می‌کنیم. و همانند فاز آموزش یک شیء با دستور \کد{cv2.face.LBPHFaceRecognizer\_create()} می‌سازیم و محتویات فایل \کد{Trainner.yml} ذخیره شده در مرحله قبل را می‌خوانیم. 

شیء دیگری برای طبقه‌بند با دستور \کد{cv2.CascadeClassifier(harcascadePath)} ایجاد می‌کنیم و با دستور \کد{cv2.VideoCapture(0)} به دوربین متصل شده به بُرد وصل می‌شویم و در حلقه‌ی بینهایت به طور متوالی فریم های دوربین را دریافت می‌کنیم و در تصویر دریافت شده ابتدا چهره«ها» را در تصویر پیدا می‌کنیم و دور آنها با عرض و ارتفاع مناسب مستطیل می‌کشیم.

سپس با استفاده از ماژول \کد{predict} چهره موجود در تصویر و ضریب اطمینان\پاورقی{Confidence } آن را که عددی بین 0 تا 100 است بدست می‌آوریم. اگر ضریب اطمینان بدست آمده بین 0 تا 50 «در این کد آستان اطمینان 50 درنظر گرفته شده است» باشد چهره به درستی شناسایی شده است. هر چقدر مقدار این آستانه را بالا تر ببرین خطا در شناسایی افزایش می‌یابد.

\شروع{شکل}[h]
\centerimg{Attendance_Test2}{15cm}
\شرح{تست سخت‌افزار و الگوریتم}
\برچسب{شکل:تست سخت}
\پایان{شکل}